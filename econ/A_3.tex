\documentclass{article}
\usepackage[margin=0.75in]{geometry}
\usepackage{amsmath}
\usepackage{amssymb}
\usepackage{graphicx}
\usepackage[thinc]{esdiff}
\usepackage{cancel}
\usepackage{titlesec}
\setlength{\parindent}{0em}
\usepackage{enumerate}

\begin{document}

\Large{
    The probability distribution with a random variable that follows the probability
    distribution function $f(x) = \frac{1}{\sigma\sqrt{2\pi}}\;e^{-\frac{1}{2}}\left(\frac{x-\mu}{\sigma}\right)^2$ is called normal 
    probability distribution.\\

    While, the distribution of a random variable that has a normal distribution with mean 0 
    and standard deviation 1 is called standard normal distribution.\\

    The advantages of a standard normal distribution are as follows -
    \begin{enumerate}[i)]
        \item The standard normal distribution has a much simpler formula 
            $f(x) = \frac{1}{\sqrt{2\pi}}\;e^{-\frac{1}{z}}(z)^2$, with a constant mean and variance
            making it easier to understand.

        \item Using the z-score, we can compare values across different scales and distributions.

        \item Using the standard normal distribution, we can get specific probabilities for values as it is constant.

        \item The standard normal variable $z=\frac{x-\mu}{\sigma}$ can be used to measure the relative location of any value of x over the distribution.
    \end{enumerate}

    Now, for the standardized normal variable $z=\frac{x-\mu}{\sigma}$\\

    \textbf{\underline{Mean:}}\\
    \begin{eqnarray*}
        E(x) &=& E(\frac{x-\mu}{\sigma})\\
        &=& \frac{1}{\sigma} E(x-\mu)\hspace{1cm}\left[E(ax) = a\cdotE(x)\right]\\
        &=& \frac{1}{\sigma}\left[E(x) - \mu\right]\\
        &=& \frac{1}{\sigma} [\mu-\mu]\hspace{1cm}\left[E(x) = \mu\right]\\
        &=& \frac{1}{\sigma}\cdot 0 \\
        &=& 0
    \end{eqnarray*}
    $\therefore E(x) = 0$\\

    \newpage
    \textbf{\underline{Standard deviation:}}\\
    We know,

    \hspace{4em}Standard deviation $\sigma = \sqrt{\sigma^2} = \sqrt{var(z)}$

    \begin{eqnarray*}
        var(z) &=& var\left[\frac{x-\mu}{\sigma}\right]\\
        &=& \frac{1}{\sigma^2}\;var(x-\mu)\hspace{1cm}\left[var(ax) = a^2\;var(x)\right]\\
        &=& \frac{1}{\sigma^2}\;\left[var(x) + var(\mu)\right]\hspace{1cm}[var(x-y) = var(x)+var(y)]\\
        &=& \frac{1}{\sigma^2}[\sigma^2 + 0]\hspace{1cm}[\mu = \text{constant}]\\
        &=& \frac{1}{\sigma^2}\cdot \sigma^2\\
        &=& 1
    \end{eqnarray*}

    $\therefore$ standard deviation, $\sigma = \sqrt{var(z)} = \sqrt{1} = 1$
    

\end{document}
