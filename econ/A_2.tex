\documentclass{article}
\usepackage[margin=0.75in]{geometry}
\usepackage{amsmath}
\usepackage{amssymb}
\usepackage{graphicx}
\usepackage[thinc]{esdiff}
\usepackage{cancel}
\usepackage{titlesec}
\setlength{\parindent}{0em}

\begin{document}

\Large{
    Given statement, ``Two mutally exclusive events are always independtent to each other" is false.\\

    Two events are mutually exclusive if the occurence of one event excludes the occurence of the other. They cannot occur simultaneously and they have no common elements. So $p(A\cap B) = 0$\\

    On the other hand two events are said to be independent if the probability of one occurence is uninfluenced by the other and vice versa.
    Mathematically, $A$ and $B$ are independent if probability of $A$ given $B$, $p(A|B) = \frac{p(A\cap B)}{p(B)}$ is equal to $p(A)$.a\\

    Let, us take an example of a two coin tosses where getting heads $H$ is one event and getting tails $T$ is other. Since, they can't occur simultaneously they are mutually exclusve, $P(H \cap T) = 0$\\

    \begin{eqnarray*}
        \text{Now, } &&p(H|T) = \frac{p(H\cap T)}{p(T)} = \frac{0}{\frac{1}{2}} = 0\hspace{10cm}\\
        &&p(H) = \frac{1}{2}\;\;\;\; \therefore p(H|T) \neq p(H)
    \end{eqnarray*}
    \\
    Thus, mutually exclusive events can be dependent. They can be independent in special case of zero probability of one event.\\

    So, the given statement is false.
}
\end{document}

