\documentclass{article}
\usepackage[margin=0.75in]{geometry}
\usepackage{amsmath}
\usepackage{amssymb}
\usepackage{graphicx}
\usepackage[thinc]{esdiff}
\usepackage{cancel}
\usepackage{titlesec}
\setlength{\parindent}{0em}
\usepackage{diagbox}

\begin{document}
\titleformat{\section}
{\Large\bfseries}
{Problem No. \thesection}
{0.5em}
{}

\titleformat{\subsection}[runin]
  {\normalfont\Large\bfseries}{(\thesubsection)}{1em}{}

\renewcommand\thesubsection{\roman {subsection}}

\section{}
\Large{
    \hspace{1.5em}The given data may be expressed in terms of binomial probability.\\

    \hspace{1.5em}Here,

    \hspace{1.5em}Success = Employed and Failure = Unemployed

    \hspace{1.5em}Probability of success, $ p = 40\;\% = \frac{40}{100} = 0.4$

    \hspace{1.5em}Number of trials, $n = 20$
    
    \subsection{}
    For $x=8$,
    \begin{eqnarray*}
        \text{probability of success, } f(x=8)  &=& {n \choose x} p^x (1-p)^{n-x}\hspace{4.5cm}\\
        &=& {20 \choose 8} 0.4^8\cdot (1-0.4)^{20-8}\\
        &=& 125970\cdot 0.4^8 \cdot 0.6^{12}\\
        &=& 0.18 \hspace{1cm}\text{(approx.)}
    \end{eqnarray*}

    \subsection{}
    Now,

    \hspace{1.5em}the probability of at least 14 being unemployed is the same as that of at most 

    \hspace{1.5em}(20-14) = 6 being employed.

    \hspace{1.5em}The probability of at most 6 being employed is the sum of the probabilities upto 

    \hspace{1.5em}x=6 from x=0,
    \begin{eqnarray*}
        \text{p(at least 14)} &=& \sum_{x=0}^6 {20 \choose x} \cdot (0.4)^x \; (0.6)^{20-x}\hspace{6.5cm}\\
        &=& {20 \choose 0} \cdot (0.4)^0\cdot(0.6)^{20} + \dots + {20 \choose 6} (0.4)^6\cdot (0.6)^{14}\\
        &=& 0.25\hspace{1cm}\text{(approx.)}
    \end{eqnarray*}
    \hspace{1.5em}The probability of at least 14 people not getting offer is 0.25

    
}
\end{document}

