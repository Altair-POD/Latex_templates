\documentclass{article}
\usepackage[margin=0.75in]{geometry}
\usepackage{amsmath}
\usepackage{amssymb}
\usepackage{titlesec}
\usepackage{graphicx}
\usepackage{enumerate}
\usepackage{tikz}
\usepackage{tkz-euclide}
\usepackage{cancel}
\usepackage[thinc]{esdiff}
\setlength{\parindent}{0em}

\begin{document}
\titleformat{\section}
{\Large\bfseries}
{\thesection.}
{0.5em}
{}[\titlerule]

\def\renum{\;\rm I\!R}
\newcommand{\ncr}[2]{\,^{#1} C _{#2}}
\newcommand{\npr}[2]{\,^{#1} P _{#2}}


\begin{minipage}[c]{0.5\linewidth}
\textbf{Injectivity:}
    $$f(x) = \frac{x+1}{x+5}$$
    Let, $a,\;b\in D_f$ such that $f(a) = f(b)$. If we can prove that a=b then the function is injective, otherwise it is not.
\begin{eqnarray*}
    f(a) &=& f(b)\\
    or,\;\frac{a+1}{a+5} &=& \frac{b+1}{b+5}\\
    or,\; ab+5a+b+5 &=& ab+5b+a+5\\
    or,\; 5a-a &=& 5b-b\\
    \therefore a&=& b
\end{eqnarray*}
$\therefore\;f(x)$  is injective.
\end{minipage}
\begin{minipage}[c]{0.3\linewidth}
\textbf{Surjectivity:}
\\let,
\begin{eqnarray*}
    f(x) &=& y\\
    x &=& f^{-1}(y)
\end{eqnarray*}
now,
\begin{eqnarray*}
    y &=& \frac{x+1}{x+5}\\
    or,\; xy +5y &=& x+1\\
    or,\; xy-x &=& 1-5y\\
    or,\; x(y-1)&=& 1-5y\\
    or,\; x &=& \frac{1-5y}{y-1}\\
    \therefore f^{-1}(x) &=& \frac{1-5x}{x-1}
\end{eqnarray*}
$\therefore \; R_f  = \renum-\{1\}$ \\
Co-domain $\ne R_f$\\
$\therefore\; f(x)$ is not surjective.
\end{minipage}



\end{document}



