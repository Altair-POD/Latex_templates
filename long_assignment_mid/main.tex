\documentclass{article}
\usepackage[margin=0.75in]{geometry}
\usepackage{amsmath}
\usepackage{amssymb}
\usepackage{titlesec}
\usepackage{graphicx}
\usepackage{enumerate}
\usepackage{tikz}
\usepackage{tkz-euclide}
\usepackage{cancel}
\usepackage[thinc]{esdiff}
\setlength{\parindent}{0em}

\begin{document}
\titleformat{\section}
{\Large\bfseries}
{\thesection.}
{0.5em}
{}[\titlerule]

\def\renum{\;\rm I\!R}
\newcommand{\ncr}[2]{\,^{#1} C _{#2}}
\newcommand{\npr}[2]{\,^{#1} P _{#2}}

\setcounter{section}{1}

\section{Problems}
\subsection{Find the Domains and the Ranges of the following functions:}
\begin{enumerate}
    \item $D_f = (-\infty, \infty)$\\$R_f= [-1, 1]$
    \item $D_f=\left[\pi n, \frac{\pi}{2}+\pi n \right)\cup \left(\frac{\pi}{2}+\pi n, \pi + \pi n\right) $\\$R_f = (-\infty, \infty)$
    \item $D_f = (-\infty, \infty)$\\$R_f = [-\sqrt{2}, \sqrt{2}]$
    \item $D_f = \renum-\{-1, 1\}$\\$R_f=\renum$
    \item $D_f = \renum - \{-\sqrt{2}, \sqrt{2}\}$\\$R_f=\left(-\infty, -\frac{1}{2}\right] \cup (0, \infty)$
    \item $D_f = (-\infty, \infty)$ \\ $R_f = (-\infty, 3]$
    \item $D_f = (-1, 1)\\$ $R_f = (-\infty, 8] $
    \item $D_f= (-\infty, \infty)$\\$R_f=[0, \infty)$
    \item $D_f= (-1, 0)\cup(0, 1)$\\$R_f=(-\infty, 0)$
    \item $D_f= \renum-\{-1, 1\}$\\$R_f = (-\infty, 0)\cup[1, \infty)$
\end{enumerate}

\subsection{Find out if the following functions are injections, surjections or bijections:}
\begin{enumerate}
\item{
\begin{minipage}[t]{0.4\linewidth}
\textbf{Injectivity:}
    $$f(x) = \frac{x+1}{x+5}$$
    Let, $a,\;b\in D_f$ such that $f(a) = f(b)$. If we can prove that a=b then the function is injective, otherwise it is not.
\begin{eqnarray*}
    f(a) &=& f(b)\\
    or,\;\frac{a+1}{a+5} &=& \frac{b+1}{b+5}\\
    or,\; ab+5a+b+5 &=& ab+5b+a+5\\
    or,\; 5a-a &=& 5b-b\\
    \therefore a&=& b
\end{eqnarray*}
$\therefore\;f(x)$  is injective.
\end{minipage}\hfill
\begin{minipage}[t]{0.4\linewidth}
\textbf{Surjectivity:}
\\let,
\begin{eqnarray*}
    f(x) &=& y\\
    x &=& f^{-1}(y)
\end{eqnarray*}
now,
\begin{eqnarray*}
    y &=& \frac{x+1}{x+5}\\
    or,\; xy +5y &=& x+1\\
    or,\; xy-x &=& 1-5y\\
    or,\; x(y-1)&=& 1-5y\\
    or,\; x &=& \frac{1-5y}{y-1}\\
    \therefore f^{-1}(x) &=& \frac{1-5x}{x-1}
\end{eqnarray*}
$\therefore \; R_f  = \renum-\{1\}$ \\
Co-domain $\ne R_f$\\
$\therefore\; f(x)$ is not surjective.
\end{minipage}

\begin{center}
{$\therefore f(x)$ is not bijective.}
\end{center}
\vspace{1cm}
}

\item{
\begin{minipage}[t]{0.4\linewidth}
\textbf{Injectivity:}
    $$f(x) = \frac{x^2 +1}{x^2+5}$$
Let, $a,\;b\in D_f$ such that $f(a) = f(b)$. If we can prove that a=b then the function is injective, otherwise it is not.
\begin{eqnarray*}
    f(a) &=& f(b\\
    or,\;\frac{a^2+1}{a^2+5} &=& \frac{b^2+1}{b^2+5}\\
    or,\; (a^2+1)(b^2+5) &=& (b^2+1)(a^2+5)\\
    or,\; 5a^2-a^2 &=& 5b^2 - b^2\\
    or,\; a^2 &=& b^2\\
    \therefore \; a &=& \pm b
\end{eqnarray*}
    $\therefore\;f(x)$ is not injective.
\end{minipage}\hfill
\begin{minipage}[t]{0.4\linewidth}
\textbf{Surjectivity:}
\\let,
\begin{eqnarray*}
    f(x) &=& y\\
    x &=& f^{-1}(y)
\end{eqnarray*}
now,
\begin{eqnarray*}
    y &=& \frac{x^2+1}{x^2+5}\\
    or,\;x^2y+5y &=& x^2+1\\
    or,\; x^2y-x^2 &=& 1-5y\\
    or,\;x^2(y-1) &=& 1-5y\\
    or,\; x &=& \sqrt{\frac{1-5y}{y-1}}
\end{eqnarray*}
$\therefore\; R_f = \left[\frac{1}{5}, 1\right)$
\\Co-domain $\ne\;R_f$\\
$\therefore\;f(x)$ is not surjective
\end{minipage}

\vspace{1cm}
\begin{center}
{$\therefore\;f(x)$ is not bijective.}
\end{center}
\vspace{1cm}
}

\item{
\begin{minipage}[t]{0.4\linewidth}
\textbf{Injectivity:}
    $$f(x) = x^7 + x^3$$
Let, $a,\;b\in D_f$ such that $f(a) = f(b)$. If we can prove that a=b then the function is injective, otherwise it is not.
\begin{eqnarray*}
    f(a) &=& f(b)\\
    or,\;a^7+a^3 &=& b^7 + b^3\\
    \therefore\; a &=& b
\end{eqnarray*}
$\therefore\;f(x)$ is injective.
\end{minipage}\hfill
\begin{minipage}[t]{0.4\linewidth}
\textbf{Surjectivity:}
\\$f(x)$ is defined for all real values of x.\\
$\therefore\; R_f = \renum$
\\Co-domain $=R_f$\\
$\therefore\;f(x)$ is surjective.
\end{minipage}
\vspace{1cm}
\begin{center}
$f(x)$ is bijective.
\end{center}
\vspace{1cm}
}

\item{

\begin{minipage}[t]{0.4\linewidth}
\textbf{Injectivity:}
    $$f(x) = x^7 + x^4$$
Let, $a,\;b\in D_f$ such that $f(a) = f(b)$. If we can prove that a=b then the function is injective, otherwise it is not.
\begin{eqnarray*}
    f(a) &=& f(b)\\
    or,\;a^7+a^4 &=& b^7 + b^4\\
    \therefore\; a &\ne& b
\end{eqnarray*}
$\therefore\;f(x)$ is not injective.
\end{minipage}\hfill
\begin{minipage}[t]{0.4\linewidth}
\textbf{Surjectivity:}
\\$f(x)$ is defined for all real values of x.\\
$\therefore\; R_f = \renum$
\\Co-domain $=R_f$\\
$\therefore\;f(x)$ is surjective.
\end{minipage}
\vspace{1cm}
\begin{center}
$f(x)$ is not bijective.
\end{center}
\vspace{1cm}
}

\item{

\begin{minipage}[t]{0.4\linewidth}
\textbf{Injectivity:}
    $$f(x) = \frac{1}{\log{(x)}-1}$$
Let, $a,\;b\in D_f$ such that $f(a) = f(b)$. If we can prove that a=b then the function is injective, otherwise it is not.
\begin{eqnarray*}
    f(a) &=& f(b)\\
    or,\;\frac{1}{\log{(a)}-1} &=& \frac{1}{\log{(b)}-1}\\
    or,\; \log(a) &=& \log(b)\\
    \therefore a&=&b
\end{eqnarray*}
    $\therefore\;f(x)$ is injective.
\end{minipage}\hfill
\begin{minipage}[t]{0.4\linewidth}
\textbf{Surjectivity:}
\begin{eqnarray*}
    y &=& \frac{1}{\log(x)-1}\\
    or,\; y\;\log(x)-y &=& 1\\
    or,\; y\;\log(x) &=& 1+y\\
    or,\; \log(x) &=& \frac{1+y}{y}\\
    \therefore x &=& 10^{\frac{1+y}{y}}
\end{eqnarray*}
    $\therefore\;R_f = (-\infty, 0)\cup(0,\infty)$\\
Co-domain $\ne R_f$\\
    $\therefore\;f(x)$is not surjective.
\end{minipage}
\vspace{1cm}
\begin{center}
$f(x)$ is not bijective.
\end{center}
\vspace{1cm}
}

\item{
\begin{minipage}[t]{0.4\linewidth}
\textbf{Injectivity:}
    $$f(x) = e^{3x}$$
Let, $a,\;b\in D_f$ such that $f(a) = f(b)$. If we can prove that a=b then the function is injective, otherwise it is not.
\begin{eqnarray*}
    f(a) &=& f(b)\\
    or,\;e^{3a} &=& e^{3b}\\
    or,\;\ln{e^{3a}} &=& \ln{e^{3b}}\\
    or,\;3a&=&3b\\
    \therefore a&=&b
\end{eqnarray*}
$\therefore\;f(x)$is injective.
\end{minipage}\hfill
\begin{minipage}[t]{0.4\linewidth}
\textbf{Surjectivity:}
\begin{eqnarray*}
    y&=&e^{3x}\\
    or,\;\ln{y} &=& \ln{e^{3x}}\\
    or,\;\ln{y} &=& 3x\\
    or,\;x &=& \frac{\ln{y}}{3}
\end{eqnarray*}
    $\therefore\;R_f=(0,\infty)$\\ 
Co-domain $\ne R_f$\\
    $\therefore\;f(x)$is not surjective.
\end{minipage}

\vspace{1cm}
\begin{center}
$\therefore\;f(x)$ is not bijective.
\end{center}
\vspace{1cm}
}
\end{enumerate}
\subsection{How many integers between 0 and 500:}
\begin{enumerate}
\item Have distinct digits $10 + 9\times 9 + 4\times9\times8$ or 379.
\item Are divisible by 3 will be 500/3 or 166.
\item{Are divisible by 3 or 5.\\
By using P.I.E we get,
\begin{eqnarray*}
    n(A) &=& 500/3 = 166\\
    n(B) &=& 500/5 = 100\\
    n(A\cap B) &=& 500/15 = 33\\
    \therefore n(A\cup B) &=& n(A) + n(B) - n(A\cap B)\\
    &=& 166 + 100 -33\\
    &=& 233
\end{eqnarray*}
}
\item{ Are divisible by either 3 or 5, but not both,\\ 
= $n(A) + n(B) - n(A\cap B)-n(A\cap B)$\\
= 166 + 100 - 33 - 33
= 200
}

\item{Are divisible by 3 but not 5,\\
= $n(A) - n(A\cap B)$\\
= 166 - 33\\
= 133
}

\end{enumerate}
\subsection{How many diagonals does a 12-sided convex polygon have? What is the general rule for a convex polygon with n sides?}
Connecting 2 points make  a diagonal but we have to omit the sides which also can be formed with 2 points.
\\So we get,
\begin{eqnarray*}
    diagonals &=& \ncr{n}{2}-n \hspace{2cm}\text{[where n is the number of sides]}\\
    &=& \ncr{12}{2} - 12\\
    &=& 54
\end{eqnarray*}
\subsection{m, n, o, p, q problem}
we can arrange m, n, o, p, q in 5! ways. If we consider (mo) and (no) as one letter and we will get something like that, (mo)npq and (no)mpq. We can arrange these in 4! ways. Now, if we remove these two permutation from the total number of permutation we will get our valid
answer that is $5! - (4!\times 2)$.
\subsection{The password problem}
\begin{enumerate}
    \item there can be $(62)^8 + (62)^9 + (62)^{10}+ (62)^{11}$ passwords for this locked door.
    \item $10^6$ permutations in 1 sec\\
        1 permutation in $\frac{1}{10^6}$ sec\\
        $\therefore$ $(62)^8 + (62)^9 + (62)^{10}+ (62)^{11}$ permutations in $\frac{(62)^8 + (62)^9 + (62)^{10}+ (62)^{11}}{10^6}$ sec


\end{enumerate}

\subsection{python variable problem}
\large{
if the variable is 1 character long then, 53\\
if the varible is 2 character long then, $53 \times 63$\\
if the varible is 3 character long the, $53\times(63)^2$\\
if the varible is 4 character long then $53\times(63)^3$\\
if the varible is 5 character long then $53\times(63)^4$\\
if the varible is 6 character long then $53\times(63)^5$\\
if the varible is 7 character long then $53\times(63)^6$\\
so, if we abide by these two rule python will have $53\left\{1+63+(63)^2+(63)^3+(63)^4 + (63)^5 +(63)^6\right\}$ varible name
}

\subsection{A coin is flipped 10 times}
\begin{enumerate}
    \item contain no heads will be 1
    \item contain exactly three heads will be $\ncr{10}{3}$
    \item at least three heads means $r\ge3$. So,
        $\ncr{10}{3}+\ncr{10}{4}+\ncr{10}{5}+\ncr{10}{6}+\ncr{10}{7}+\ncr{10}{8}+\ncr{10}{9}+\ncr{10}{10}$
    \item contain more heads than tails means $r\ge6$. So, $\ncr{10}{6}+\ncr{10}{7}+\ncr{10}{8}+\ncr{10}{9}+\ncr{10}{10}$
\end{enumerate}

\subsection{coefficient of $x^{28}$ in $(2x^3 - x)^{16}$}
\begin{eqnarray*}
    \ncr{n}{r}\cdot(a)^{n-r}\cdot{(b)}^r &=& \ncr{16}{r}\cdot(2x^3)^{16-r}\cdot(-x)^r\\
    &=& \ncr{16}{r}\cdot 2^{16-r}\cdot x^{48-3r}\cdot (-1)^r \cdot x^r\\
    &=& \ncr{16}{r}\cdot 2^{16-r} \cdot (-1)^{r} \cdot x^{48-2r}
\end{eqnarray*}

if we compare,
\begin{eqnarray*}
    x^{48-2r} &=& x^{28}\\
    2r &=& 48-28\\
    2r &=& 20\\
    \therefore r&=& 10
\end{eqnarray*}

the coefficient of $x^{28}$ is $\ncr{16}{10}\cdot 2^{6}$

\subsection{coefficient of $x^2$ in $\left(3x-\frac{2}{x^2}\right)^{23}$}

\begin{eqnarray*}
    \ncr{n}{r}\cdot (a)^{n-r} \cdot (b)^r &=& \ncr{23}{r}\cdot (3x)^{23-r}\cdot \left(-\frac{2}{x^2}\right)^r\\
    &=& \ncr{23}{r}\cdot 3^{23-r}\cdot x^{23-r}\cdot (-2)^r \cdot x^{-2r}\\
    &=& \ncr{23}{r} \cdot 3^{23-r}\cdot (-2)^r\cdot x^{23-3r}
\end{eqnarray*}

if we compare,
\begin{eqnarray*}
    x^{23-3r} &=& x^2\\
    23-3r &=& 2\\
    23-2 &=& 3r\\
    \therefore r &=& 7
\end{eqnarray*}


the coefficient of $x^2$ is $\ncr{23}{7}\cdot (3)^{16}\cdot (-2)^7$

\subsection{How many solutions are there to the equation: $x_1 +x_2+x_3 = 19$}
\begin{enumerate}
    \item $x_1$, $x_2$, $x_3$ are all non-negative integers:\\
        from the number of Positive integral solutions formula for non-negative integers we get,
        $$\ncr{n+r-1}{r-1}$$\\
    here, $n$ is 19 and $r$ is the number of variable in the equation that is 3.
        So there will be $\ncr{19+3-1}{3-1}$ or $\ncr{21}{2}$ or 210 solutions for all non-negative integers.
    \item $x_1$, $x_2$, $x_3$ are all positive integers:
        from the number of positive integral solutions for positive integers we get,
        $$\ncr{n-1}{r-1}$$
        here, $n$ is 19 and $r$ is the number of variable in the equation that is 3. So there be $\ncr{19-1}{3-1}$ or $\ncr{18}{2}$ or 153 solutions for all positive integers.
\end{enumerate}

\subsection{ATTRACTION}
If we gather all the $T$s together; we will find something like this $AA(TTT)RCION$. The permutation for this will be $\frac{7!}{2!}$.\\
Now, if we remove $(TTT)$ from $AA(TTT)RCION$ we will get $AARCION$; in here we can see 6 gaps between each letter also we have 1 gap in the hard right and 1 gap in the hard left; that sums up to 8 gaps altogether. \\
We can put each $T$ in any of this 8 gaps. So, the permutation of this will be $\npr{8}{3}$. Also, these $T$s will also have their own permutation between them so the final permutation will be $\frac{\npr{8}{3}}{3!}$ which brings us to $\ncr{8}{3}$.\\
So, the answer is $\frac{7!}{2!}\times \ncr{8}{3}$

\subsection{Jiminy Cricket}

For Jiminy Cricket to go from(0,0) to (10, 6). He has to take 10 steps to right and 6 steps to upward or he has to take total 16 steps to reach the destination.  

So, there will be 16 steps in each way for it to go from (0,0) to (10,6). But we have to consider only the distinct permutations that means we have to omit all the similar permutations that can occur.\\
So the answer is $\frac{16!}{10!\cdot 6!}$


\subsection{Playing Twenty-Nine}
There are 32 cards in total.\\
Since we are distributing 32 cards among 4 players; so each one of them will get 32/4 or 8 cards.
The situation is dependent.\\
So, the total ways will be $= \ncr{32}{8}\times\ncr{24}{8}\times\ncr{16}{8}\times\ncr{8}{8}$


\end{document}


