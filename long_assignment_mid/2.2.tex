\documentclass{article}
\usepackage[margin=0.75in]{geometry}
\usepackage{amsmath}
\usepackage{amssymb}
\usepackage{titlesec}
\usepackage{graphicx}
\usepackage{enumerate}
\usepackage{tikz}
\usepackage{tkz-euclide}
\usepackage{cancel}
\usepackage[thinc]{esdiff}
\usepackage{parallel}
\setlength{\parindent}{0em}

\begin{document}
\titleformat{\section}
{\Large\bfseries}
{\thesection.}
{0.5em}
{}[\titlerule]

\def\renum{\;\rm I\!R}
\newcommand{\ncr}[2]{\,^{#1} C _{#2}}
\newcommand{\npr}[2]{\,^{#1} P _{#2}}

\begin{minipage}[t]{0.4\linewidth}
\textbf{Injectivity:}
    $$f(x) = \frac{x+1}{x+5}$$
    Let, $a,\;b\in D_f$ such that $f(a) = f(b)$. If we can prove that a=b then the function is injective, otherwise it is not.
\begin{eqnarray*}
    f(a) &=& f(b)\\
    or,\;\frac{a+1}{a+5} &=& \frac{b+1}{b+5}\\
    or,\; ab+5a+b+5 &=& ab+5b+a+5\\
    or,\; 5a-a &=& 5b-b\\
    \therefore a&=& b
\end{eqnarray*}
$\therefore\;f(x)$  is injective.
\end{minipage}
\hfill
\begin{minipage}[t]{0.4\linewidth}
\textbf{Surjectivity:}
\\let,
\begin{eqnarray*}
    f(x) &=& y\\
    x &=& f^{-1}(y)
\end{eqnarray*}
now,
\begin{eqnarray*}
    y &=& \frac{x+1}{x+5}\\
    or,\; xy +5y &=& x+1\\
    or,\; xy-x &=& 1-5y\\
    or,\; x(y-1)&=& 1-5y\\
    or,\; x &=& \frac{1-5y}{y-1}\\
    \therefore f^{-1}(x) &=& \frac{1-5x}{x-1}
\end{eqnarray*}
$\therefore \; R_f  = \renum-\{1\}$ \\
Co-domain $\ne R_f$\\
$\therefore\; f(x)$ is not surjective.
\end{minipage}

\vspace{1cm}
\center{$\therefore f(x)$ is not bijective.}
\vspace{1cm}

\begin{minipage}[t]{0.4\linewidth}
\textbf{Injectivity:}
    $$f(x) = \frac{x^2 +1}{x^2+5}$$
Let, $a,\;b\in D_f$ such that $f(a) = f(b)$. If we can prove that a=b then the function is injective, otherwise it is not.
\begin{eqnarray*}
    f(a) &=& f(b\\
    or,\;\frac{a^2+1}{a^2+5} &=& \frac{b^2+1}{b^2+5}\\
    or,\; (a^2+1)(b^2+5) &=& (b^2+1)(a^2+5)\\
    or,\; 5a^2-a^2 &=& 5b^2 - b^2\\
    or,\; a^2 &=& b^2\\
    \therefore \; a &=& \pm b
\end{eqnarray*}
    $\therefore\;f(x)$ is not injective.
\end{minipage}\hfill
\begin{minipage}[t]{0.4\linewidth}
\textbf{Surjectivity:}
\\let,
\begin{eqnarray*}
    f(x) &=& y\\
    x &=& f^{-1}(y)
\end{eqnarray*}
now,
\begin{eqnarray*}
    y &=& \frac{x^2+1}{x^2+5}\\
    or,\;x^2y+5y &=& x^2+1\\
    or,\; x^2y-x^2 &=& 1-5y\\
    or,\;x^2(y-1) &=& 1-5y\\
    or,\; x &=& \sqrt{\frac{1-5y}{y-1}}
\end{eqnarray*}
$\therefore\; R_f = \left[\frac{1}{5}, 1\right)$
\\Co-domain $\ne\;R_f$
$\therefore\;f(x)$ is not surjective
\end{minipage}
\vspace{1cm}
\center{$\therefore\;f(x)$is not bijective.}
\vspace{1cm}
\newpage

\begin{minipage}[t]{0.4\linewidth}
\textbf{Injectivity:}
    $$f(x) = x^7 + x^3$$
Let, $a,\;b\in D_f$ such that $f(a) = f(b)$. If we can prove that a=b then the function is injective, otherwise it is not.
\begin{eqnarray*}
    f(a) &=& f(b)\\
    or,\;a^7+a^3 &=& b^7 + b^3\\
    \therefore\; a &=& b
\end{eqnarray*}
$\therefore\;f(x)$ is injective.
\end{minipage}\hfill
\begin{minipage}[t]{0.4\linewidth}
\textbf{Surjectivity:}
\\$f(x)$ is defined for all real values of x.\\
$\therefore\; R_f = \renum$
\\Co-domain $=R_f$\\
$\therefore\;f(x)$ is surjective.
\end{minipage}
\vspace{1cm}
\center{$f(x)$ is bijective.}
\vspace{1cm}

\begin{minipage}[t]{0.4\linewidth}
\textbf{Injectivity:}
    $$f(x) = x^7 + x^4$$
Let, $a,\;b\in D_f$ such that $f(a) = f(b)$. If we can prove that a=b then the function is injective, otherwise it is not.
\begin{eqnarray*}
    f(a) &=& f(b)\\
    or,\;a^7+a^4 &=& b^7 + b^4\\
    \therefore\; a &=& b
\end{eqnarray*}
$\therefore\;f(x)$ is injective.
\end{minipage}\hfill
\begin{minipage}[t]{0.4\linewidth}
\textbf{Surjectivity:}
\\$f(x)$ is defined for all real values of x.\\
$\therefore\; R_f = \renum$
\\Co-domain $=R_f$\\
$\therefore\;f(x)$ is surjective.
\end{minipage}
\vspace{1cm}
\center{$f(x)$ is bijective.}
\vspace{1cm}
\newpage

\begin{minipage}[t]{0.4\linewidth}
\textbf{Injectivity:}
    $$f(x) = \frac{1}{\log{(x)}-1}$$
Let, $a,\;b\in D_f$ such that $f(a) = f(b)$. If we can prove that a=b then the function is injective, otherwise it is not.
\begin{eqnarray*}
    f(a) &=& f(b)\\
    or,\;\frac{1}{\log{(a)}-1} &=& \frac{1}{\log{(b)}-1}\\
    or,\; \log(a) &=& \log(b)\\
    \therefore a&=&b
\end{eqnarray*}
    $\therefore\;f(x)$ is injective.
\end{minipage}\hfill
\begin{minipage}[t]{0.4\linewidth}
\textbf{Surjectivity:}
\begin{eqnarray*}
    y &=& \frac{1}{\log(x)-1}\\
    or,\; y\;\log(x)-y &=& 1\\
    or,\; y\;\log(x) &=& 1+y\\
    or,\; \log(x) &=& \frac{1+y}{y}\\
    \therefore x &=& 10^{\frac{1+y}{y}}
\end{eqnarray*}
    $\therefore\;R_f = (-\infty, 0)\cup(0,\infty)$\\
Co-domain $\ne R_f$\\
    $\therefore\;f(x)$is not surjective.
\end{minipage}
\vspace{1cm}
\center{$f(x)$ is not bijective.}
\vspace{1cm}

\begin{minipage}[t]{0.4\linewidth}
\textbf{Injectivity:}
    $$f(x) = e^{3x}$$
Let, $a,\;b\in D_f$ such that $f(a) = f(b)$. If we can prove that a=b then the function is injective, otherwise it is not.
\begin{eqnarray*}
    f(a) &=& f(b)\\
    or,\;e^{3a} &=& e^{3b}\\
    or,\;\ln{e^{3a}} &=& \ln{e^{3b}}\\
    or,\;3a&=&3b\\
    \therefore a&=&b
\end{eqnarray*}
$\therefore\;f(x)$is injective.
\end{minipage}\hfill
\begin{minipage}[t]{0.4\linewidth}
\textbf{Surjectivity:}
\begin{eqnarray*}
    y&=&e^{3x}\\
    or,\;\ln{y} &=& \ln{e^{3x}}\\
    or,\;\ln{y} &=& 3x\\
    or,\;x &=& \frac{\ln{y}}{3}
\end{eqnarray*}
    $\therefore\;R_f=(0,\infty)$\\ 
Co-domain $\ne R_f$\\
    $\therefore\;f(x)$is not surjective.
\end{minipage}

\vspace{1cm}
\center{$\therefore\;f(x)$ is not bijective.}
\vspace{1cm}


\end{document}


